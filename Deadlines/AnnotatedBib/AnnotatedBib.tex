\documentclass[12pt]{article}
\usepackage[margin=0.1in]{geometry}
\usepackage{xcolor}
\usepackage{framed}
\usepackage{enumitem}
\usepackage{natbib}
\usepackage{mathtools,xparse}
\usepackage{graphics}
\colorlet{shadecolor}{orange!15}
% \definecolor{shadecolor}{rgb}{255,128,0}\
\usepackage{float}
\usepackage{fullpage} % Package to use full page
\usepackage{parskip} % Package to tweak paragraph skipping
\usepackage{tikz} % Package for drawing
\usepackage{amsmath}
\usepackage{amssymb}
\usepackage{hyperref}
\usepackage{setspace}
\usepackage{graphicx} % Allows including images
\usepackage{booktabs} % Allows the use of \toprule, \midrule and \bottomrule in tables
\usepackage{longtable}
\usepackage{indentfirst}
\usepackage{tikz}
\usetikzlibrary{arrows,shapes,positioning,shadows,trees}

\tikzset{
  basic/.style  = {draw, text width=2cm, drop shadow, font=\sffamily, rectangle},
  root/.style   = {basic, rounded corners=2pt, thin, align=center,
                   fill=green!30},
  level 2/.style = {basic, rounded corners=6pt, thin,align=center, fill=green!60,
                   text width=8em},
  level 3/.style = {basic, thin, align=left, fill=pink!60, text width=6.5em}
}

%Allows multi-column tables 
\input{../tcilatex}

\setlength{\topmargin}{-0.4in} 
\setlength{\textheight}{8.9in}
\setlength{\parindent}{2em}
% \setstretch{1.25}
\doublespacing
\title{\singlespacing Annotated bibliography}
\author{Rachel Anderson\thanks{Mailing Address: Department of Economics, Julis Romo Rabinowitz Building,
Princeton, NJ 08544. Email: rachelsa@Princeton.edu.
This project received funding from the NIH (Grant 1 R01 HD077227-01). }}
\date{This Version: \today}

\begin{document}

\maketitle


\section*{(Meta) Matching�Papers}
\cite{bailey2017} review literature on historical record linkage in US and examines performance of automated record linkage algorithms with two high-quality historical datasets and one synthetic ground truth.  They conclude that no method consistently produces representative samples; machine linking has high number of false links and may introduce bias into analyses.  

\section*{Matching Methods}
\begin{itemize}
\item Neter, Maynes, and Ramanathan (1965): small mismatch errors in finite population sampling can lead to a substantial bias in estimating the relationship between response errors and true values
\item Scheuren and Winkler (1993): propose method for adjusting for bias of mismatch error in OLS
\item SW (1997, 1991): iterative procedure that modifies regression and matching results for apparent outliers 
\item Lahiri and Larsen (2005):   provides unbiased estimator directly instead of bias correction for OLS, by applying regression to transformed model 
\item Abramitzky, Mill, P\'erez (2019): guide for researchers in the choice of which variables to use for linking, how to estimate probabilities, and then choose which records to use in the analysis.  Created R code and stata command to implement the method
\item Ferrie 1996, Abramitzky, BOustan and Eriksson (2012 2014 2017) are deterministic.  Conservative methods require no other potential match with same name within a 5-year band
\item Semi-automated Feigenbaum, Ruggles et al 
\item Abramitzky, Boustan, Eriksson, Feigenbaum, P\`erez (2019): evaluate different automated methods for record linkage, specifically deterministic (like Ferrie and ABE papers), machine learning Feigenbaum approach, and the AMP approach with the EM algorithm.  Document a frontier between type I and type II errors; cost of low false positive rates comes at cost of designating relatively fewer (true) matches.  Humans typically match more at a cost of more false positives.  They study how different linking methods affect inference -- sensitivity of regression estimates to the choice of linking algorithm.  They find that the parameter estimates are stable across linking methods.  Find effect of matching algorithm on inference is small. 
\item 

Treatment of equally likely -- equal probability weighting of tied candidates (Bleakley and Ferrie 2016); weighted combo of linking features to ehlp disambiguate potential matches.  Ferrie 96; Old ABE, new ABE, and Feigenbaum. 


\item Survey paper from handbook of econometrics
\item For example, Goeken et al. (2017) document that
in two enumerations of St. Louis in the 1880 Census, nearly 46 percent of first names are not exact matches,
and the Early Indicators project notes that 11.5 percent of individuals in the Oldest Old sample have a
shorter first name in pension records than in the original Civil War enlistment records (Costa et al. 2017). 
\end{itemize}

\section*{Estimation Papers}

\section*{Important Applications}


\bibliography{../proposal_bib} 
\bibliographystyle{IEEEtranN}
\end{document}