\documentclass[12pt]{article}
\usepackage[margin=0.1in]{geometry}
\usepackage{xcolor}
\usepackage{framed}
\usepackage{enumitem}
\usepackage{natbib}
\usepackage{mathtools,xparse}
\usepackage{graphics}
\colorlet{shadecolor}{orange!15}
% \definecolor{shadecolor}{rgb}{255,128,0}\
\usepackage{float}
\usepackage{fullpage} % Package to use full page
\usepackage{parskip} % Package to tweak paragraph skipping
\usepackage{tikz} % Package for drawing
\usepackage{amsmath}
\usepackage{amssymb}
\usepackage{hyperref}
\usepackage{setspace}
\usepackage{graphicx} % Allows including images
\usepackage{booktabs} % Allows the use of \toprule, \midrule and \bottomrule in tables
\usepackage{longtable}
\usepackage{indentfirst}
\usepackage{tikz}
\usetikzlibrary{arrows,shapes,positioning,shadows,trees}

\tikzset{
  basic/.style  = {draw, text width=2cm, drop shadow, font=\sffamily, rectangle},
  root/.style   = {basic, rounded corners=2pt, thin, align=center,
                   fill=green!30},
  level 2/.style = {basic, rounded corners=6pt, thin,align=center, fill=green!60,
                   text width=8em},
  level 3/.style = {basic, thin, align=left, fill=pink!60, text width=6.5em}
}

%Allows multi-column tables 
\input{../tcilatex}

\setlength{\topmargin}{-0.4in} 
\setlength{\textheight}{8.9in}
\setlength{\parindent}{2em}
% \setstretch{1.25}
\doublespacing
\title{\singlespacing The effects of matching algorithms and estimation methods using linked data}
\author{Rachel Anderson\thanks{Mailing Address: Department of Economics, Julis Romo Rabinowitz Building,
Princeton, NJ 08544. Email: rachelsa@Princeton.edu.
This project received funding from the NIH (Grant 1 R01 HD077227-01). }}
\date{This Version: \today}

\begin{document}

\maketitle


\begin{abstract}
\singlespacing
\noindent This paper studies the effect of different matching algorithms and estimation procedures for linked data on the quality of the estimates that they produce.  \end{abstract}

\section{Introduction}
In applied microeconomics, identifying a common set of individuals appearing in two or more datasets is often complicated by the absence of unique identifying variables. For example,  \cite{aizer2016} link children listed on their mother's welfare program applications with their death records using individuals' names and dates of birth.  However, since name and date combinations are not necessarily unique (and may be prone to typographical error), the authors identify cases where multiple death records seem to refer to the same individual.  Instead of dropping these observations from their analysis, they use estimation techniques from \cite{ahl2019} that allow for observations to have multiple linked outcomes.  

The methods in \cite{ahl2019} assume that each of the linked outcomes is equally likely to be the true match; however, the authors describe how to construct more efficient estimators if additional information about match quality is available.  Specifically, if the researcher can estimate the probability that each individual-outcome pair is a true match, then this knowledge can be used to achieve a reduction in mean-squared error.  Such probabilities are outputted by probabilistic record linkage procedures, first developed by \cite{fellegi69} in the statistics literature, but only recently applied to economics \cite{arp2018}. Hence, any discussion of best practices for using linked data should address how to choose matching algorithms and estimation procedures jointly.  

The goal of this paper is therefore to study the effects of different combinations of matching algorithms and estimation procedures for linked data on the quality of the estimates that they produce.  First, I will compare how different matching algorithms perform in terms of the representativeness of the matched data they produce and their tolerance for type I and type II errors.  Next, with multiple matched versions of the data in hand, I will compute point estimates and confidence intervals for the same parameter of interest using methods that vary by whether they allow for multiple matches, incorporate the matching probabilities, and are likelihood-based in their approach.  In total, I will perform the above analysis twice -- with simulated data and with real data that the simulated data are generated to imitate.  

To the best of my knowledge, how data pre-processing impacts subsequent inference in economics research is not well understood.  This paper adds to a recent series of papers by  \cite{bailey2017} and \cite{arp2018, abe2019}, which evaluate the performance of common matching algorithms for historical data in a variety of real and simulated data settings, and offer informal discussions about the impact of matching on subsequent inference.  This paper pushes these ideas further, by measuring the \textit{joint} effects of matching and estimation; with a focus on developing estimation techniques that incorporate information from the matching process to improve efficiency and accurately reflect uncertainty. 

The matching techniques that will be studied in this paper include deterministic record linkage procedures developed by \cite{ferrie96} and \cite{abe2012, abe2014, abe2017}, and \cite{aizer2016}, and multiple implementations of probabilistic record linkage, specifically the fastLink \cite{enamorado2019}, and machine learning approaches \cite{Feigenbaum2016AML}. Estimation techniques will include \cite{ahl2019}, \cite{lahiri05}, and a fully Bayesian approach that I will develop in this paper. 

The data used in this paper consist of the unmerged files from \cite{aizer2016}, which I will pre-process using the practices developed by \cite{arp2018}.  The parameter of interest is the average treatment effect of receiving a cash transfer on the long-term outcomes of children in poor families.   

Section 2 describes the general problem that this paper seeks to address.  Section 3 provides an overview of the matching techniques that I will study.  Section 4 describes the estimation techniques.  Section 5 will have simulations and results.  Section 6 will have real data and results. 

\section{General problem}
Suppose that the researcher would like to estimate $\theta_0$ in a parametric model of the form
\begin{equation} E[m(\mathbf{z_i}; \theta_0)] = 0 \label{model} \end{equation}
where $m(\cdot)$ is a moment function and $\mathbf{z_i} = (x_i, y_i)$ is the data associated with an individual $i$ sampled at random from the population of interest; however, instead of observing $(x_i, y_i)$ pairs directly, the researcher observes two datasets.  The first dataset contains variables $x_i$ and identifiers $w_i$ for individuals $i = 1, \dots, N_0$ recorded at time 0.  The second dataset contains outcomes $y_j$ and identifiers $w_j$ for individuals $j=1,\dots, N_T$, observed at time $T$ far in the future.  

To estimate ($\ref{model}$) with standard econometric methods, the researcher must identify which of the $x_i$ and $y_j$ refer to the same individuals.   That is, she needs to recover the matching function $\varphi: \{1,\dots, N_0\} \to \{1,\dots, N_T\}$, where $\varphi(i) = j$ if individual $i$ observed at time 0 and individual $j$ observed at time $T$ refer to the same entity.  If $w_i$ and $w_j$ identify individuals uniquely and do not change over time, then $\varphi(i) = j$ if and only if $w_i = w_j$.  However, if the identifiers are non-unique or prone to errors introduced by the record-generating process, then $\varphi$ needs to be estimated, and inference about $\theta$ needs to be adjusted accordingly.   

In statistics, the task of recovering $\varphi$ is called \textit{record linkage}.  A record linkage procedure consists of a set of decisions about (i) selecting and standardizing the identifiers $w_i$ and $w_j$, (ii) choosing which records to consider as potential matches (especially when $N_0 \times N_T \times \text{dim}(w_i) $ is large), (iii) defining what patterns of $(w_i,w_j)$ constitute (partial) agreements, and (iv) designating $(i,j)$ pairs as matches.  Each step of the record linkage process introduces the possibility that a true match is overlooked (Type II error), or that a false match is designated as correct (Type I error).  [ \cite{abe2019} illustrate a tradeoff between the two...].  

To fix ideas, suppose that $\theta_0$ is the long-run impact of cash transfers on the longevity of children raised in poor families.  The vector $x_i$ includes family and child characteristics as observed in welfare program applications; and the outcomes $y_j$ are constructed by calculating (day of death $-$ day of birth) for all of the observations in a set of death records.  For all $i$ and $j$, the identifiers $w_i$ and $w_j$ include the individual's first and last name, and date of birth.  Additionally, $w_i$ includes $i$'s place of birth; $w_j$ includes $j$'s place of death; and some $w_i$ and $w_j$ contain the individual's middle name or middle initial. 

In this setting, an example of a (deterministic) record linkage procedure consists of: 
\begin{enumerate}
\item[(i)] using a phonetic algorithm to standardize all string variables; 
\item[(ii)] considering as potential matches only $(i,j)$ pairs whose phonetically standardized names begin with the same letter, and whose birth years are within $\pm$2 years;
\item[(iii)] measuring agreements among names using Jaro-Winkler string distances, and weighing disagreements in birth year more than differences in birth month (and more than differences in birth day), 
\item[(iv)] designating as matches all $(i,j)$ pairs with scores calculated using the metrics in (iii) exceeding a pre-specified cutoff; and, if a record $i$ has multiple possible matches $j$ that exceed the cut-off, then choosing the match with the highest score (or picking a random match if there is a tie).  
\end{enumerate}

Another record linkage procedure could be defined using the same rules for steps (i)-(iii), but replace (iv) with a probabilistic matching rule that does not enforce one-to-one matching:
\begin{enumerate}
\item[(iv*)]  use the Expectation Maximization algorithm to estimate the probability that each record pair is a match, and then  designate matches using the \cite{fellegi69} rule for pre-specified tolerances for Type I and Type II errors 
\end{enumerate} 

Except in rare cases, the matchings produced by the preceding record linkage procedures will be distinct.  Whereas the first procedure associates each $x_i$ with at most one matched $y_j$, the second procedure may associate the same $x_i$ with multiple $y_j$ (or zero!).  In the former case, estimation proceeds as a standard GMM problem; in the latter, estimating $\theta$ requires methods that allow $\varphi$ to take on multiple values \citep{ahl2019}.  This illustration shows that not only are the point estimates of $\theta$ likely to depend on the estimates of $\varphi$, but also the methods for estimating $\theta$ are likely to be different. 

As observed in \cite{bailey2017}, record linkage procedures differ by the set of assumptions that motivate their use.  However, all of the procedures discussed in this paper will be studied under the following, common set of assumptions:
\begin{enumerate}
\item (De-duplication) Within a given dataset, each observation refers to a distinct entity.  That is, if two observations share the same identifier, they represent two different individuals.
\item (No unobserved sample selection) The observed $x_i$ and $y_j$ are random samples conditional on $w_i$ and $w_j$, respectively.  This means that all individuals with the same identifying information have equal probability of appearing in the sample. 
\item There exists a unique $\theta_0 \in \Theta$ that satisfies the relationship in (\ref{model}), that can be consistently estimated using standard econometric techniques if $\varphi_0$ is known.
\end{enumerate}

The next section discusses in detail the exact record linkage techniques that will be studied, and their motivating assumptions.

\bibliography{./proposal_bib} 
\bibliographystyle{IEEEtranN}

\end{document}