\documentclass[12pt]{article}
\usepackage[margin=0.1in]{geometry}
\usepackage{xcolor}
\usepackage{framed}
\usepackage{enumitem}
\usepackage{natbib}
\usepackage{mathtools,xparse}
\usepackage{graphics}
\colorlet{shadecolor}{orange!15}
% \definecolor{shadecolor}{rgb}{255,128,0}\
\usepackage{float}
\usepackage{fullpage} % Package to use full page
\usepackage{parskip} % Package to tweak paragraph skipping
\usepackage{tikz} % Package for drawing
\usepackage{amsmath}
\usepackage{amssymb}
\usepackage{hyperref}
\usepackage{setspace}
\usepackage{graphicx} % Allows including images
\usepackage{booktabs} % Allows the use of \toprule, \midrule and \bottomrule in tables
\usepackage{longtable}
\usepackage{indentfirst}
\usepackage{tikz}
\usetikzlibrary{arrows,shapes,positioning,shadows,trees}

\tikzset{
  basic/.style  = {draw, text width=2cm, drop shadow, font=\sffamily, rectangle},
  root/.style   = {basic, rounded corners=2pt, thin, align=center,
                   fill=green!30},
  level 2/.style = {basic, rounded corners=6pt, thin,align=center, fill=green!60,
                   text width=8em},
  level 3/.style = {basic, thin, align=left, fill=pink!60, text width=6.5em}
}

%Allows multi-column tables 
\input{../tcilatex}

\setlength{\topmargin}{-0.4in} 
\setlength{\textheight}{8.9in}
\setlength{\parindent}{2em}
% \setstretch{1.25}
\doublespacing
\title{\singlespacing The effects of matching algorithms and estimation methods using linked data}
\author{Rachel Anderson\thanks{Mailing Address: Department of Economics, Julis Romo Rabinowitz Building,
Princeton, NJ 08544. Email: rachelsa@Princeton.edu.
This project received funding from the NIH (Grant 1 R01 HD077227-01). }}
\date{This Version: \today}

\begin{document}

\maketitle


\begin{abstract}
\singlespacing
\noindent This paper studies the effect of different matching algorithms and estimation procedures for linked data on the quality of the estimates that they produce.  \end{abstract}

\section{Introduction}
In applied microeconomics, identifying a common set of individuals appearing in two or more datasets is often complicated by the absence of unique identifying variables. For example,  \cite{aizer2016} link children listed on their mother's welfare program applications with their death records using individuals' names and dates of birth.  However, since name and date combinations are not necessarily unique (and may be prone to typographical error), the authors identify cases where multiple death records seem to refer to the same individual.  Instead of dropping these observations from their analysis, they use estimation techniques from \cite{ahl2019} that allow for observations to have multiple linked outcomes.  

The methods in \cite{ahl2019} assume that each of the linked outcomes is equally likely to be the true match; however, the authors describe how to construct more efficient estimators if additional information about match quality is available.  Specifically, if the researcher can estimate the probability that each individual-outcome pair is a true match, then this knowledge can be used to achieve a reduction in mean-squared error.  Such probabilities are outputted by probabilistic record linkage procedures, first developed by \cite{fellegi69} in the statistics literature, but only recently applied to economics \cite{arp2018}. Hence, any discussion of best practices for using linked data should address how to choose matching algorithms and estimation procedures jointly.  

The goal of this paper is therefore to study the effects of different combinations of matching algorithms and estimation procedures for linked data on the quality of the estimates that they produce.  First, I will compare how different matching algorithms perform in terms of the representativeness of the matched data they produce and their tolerance for type I and type II errors.  Next, with multiple matched versions of the data in hand, I will compute point estimates and confidence intervals for the same parameter of interest using methods that vary by whether they allow for multiple matches, incorporate the matching probabilities, and are likelihood-based in their approach.  In total, I will perform the above analysis twice -- with simulated data and with real data that the simulated data are generated to imitate.  

To the best of my knowledge, how data pre-processing impacts subsequent inference in economics research is not well understood.  This paper adds to a recent series of papers by  \cite{bailey2017} and \cite{arp2018, abe2019}, which compare how common matching algorithms for historical data produce different datasets, and offer informal discussion of the impact on inferences.  This paper goes a step further, by testing also the effect of different estimation techniques that incorporate information from the matching process, and uses simulations to make more generalizable conclusions.  

Matching techniques will include deterministic record linkage procedures developed by \cite{ferrie96} and \cite{abe2012, abe2014, abe2017}, and \cite{aizer2016}, and multiple implementations of probabilistic record linkage, specifically the fastLink \cite{enamorado2019}, and machine learning approaches \cite{Feigenbaum2016AML}. Estimation techniques will include \cite{ahl2019}, \cite{lahiri05}, and a fully Bayesian approach that I will develop in this paper. 

The real data consists of the unmerged files from \cite{aizer2016}, which I will pre-process using the practices developed by \cite{arp2018}.  The parameter of interest is the average treatment effect of a conditional cash transfer program on recipients' children's longevity.   

By Friday, I will write a description of the model, the parameter of interest, and the set of assumptions that I will use.   I will also provide a list of the matching and estimation techniques (with descriptions) that I will study, as well as a timeline for implementing each of them. 

This paper is closest in spirit to \cite{bailey2017}, \cite{arp2018}, and \cite{abe2019}, which compare how different matching techniques vary in terms of representativeness, type I and type II errors.  This paper 


May be the case that if you allow for multiple matches, deterministic performs just as well as probabilstic linking! 

\section{General problem}
Suppose that the researcher would like to answer a question about life cycle dynamics, such as measuring the effect of a conditional cash transfer during a person's childhood on their expected longevity.  Specifically, she would like to estimate a parametric model of the form
\begin{equation} E[m(\mathbf{z_i}; \theta_0)] = 0 \label{model} \end{equation}
where $m(\cdot)$ is a moment function and $\mathbf{z_i} = (x_i, y_i)$ is the data associated with an individual $i$ sampled at random from the population of interest.  

Suppose that instead of observing $(x_i, y_i)$ pairs directly, the researcher observes two datasets, recorded at distinct times.  The first dataset contains variables $x_i$ and identifiers $w_i$ for individuals $i = 1, \dots, N_0$ recorded at time 0.  The second dataset contains outcomes $y_j$ and identifiers $w_j$ for individuals $j=1,\dots, N_T$, observed at time $T$ far in the future.  

To estimate ($\ref{model}$) with standard econometric methods, the researcher needs to recover the matching function $\varphi: \{1,\dots, N_0\} \to \{1,\dots, N_T\}$, such that if $\varphi(i) = j$, then individual $i$ observed at time 0 and individual $j$ observed at time $T$ refer to the same entity.  If $w_i$ and $w_j$ identify individuals uniquely and do not change over time, then $\varphi(i) = j$ if $w_i = w_j$.  However, if the identifiers are non-unique or prone to error, then $\varphi$ needs to be estimated, and it is unclear how uncertainty about $\varphi$ should impact estimates of $\theta_0$.  

Formally, the methods used to recover $\varphi$ are called a record linkage procedure.  A record linkage procedure consists of a set of decisions about (i) selecting and standardizing the identifiers $w_i$, (ii) defining what patterns of $(w_i,w_j)$ pairs constitute (partial) agreements, (iii) implementing shortcuts for computational feasibility, and (iv) designating which observation pairs qualify as matches.   Since real data are messy, each of these steps has the potential to introduces bias by ignoring certain match possibilities or understating uncertainty about $\varphi$.  The goal of this paper is to ascertain the relative importance of each of these biases for a variety of record linkage procedures, and generally develop best practices for handling the nuisance parameter $\varphi$ when the goal is estimating $\theta$.  

To fix ideas, suppose that $\theta_0$ is the true long-run impact of cash transfers on the longevity of children raised in poor families.  The variables $x_i$ include family and child characteristics as observed in welfare program applications; and the outcomes $y_j$ are constructed by calculating (day of death $-$ day of birth) for all of the individuals observed in a set of death records.  Each $x_i$ and $y_j$ are associated with first and last names and a date of birth.  Additionally, some observations have middle initials, and the identifiers $w_i$ includes place of birth, while the identifiers $w_j$ include place of death.  

In this case, an example of a (deterministic) record linkage procedure is: 
\begin{enumerate}
\item[(i)] match observations using phonetic spellings of first and last names, birth dates, and middle initials when available; 
\item[(ii)] measure agreements between names using Jaro-Winkler string distances, and penalize disagreements in birth year more than differences in birth month;  
\item[(iii)] consider as potential matches only $(i,j)$ pairs whose (phonetically standardized) initials match, and whose birth years are within $\pm$2 years;
\item[(iv)] accept as matches all $(i,j)$ pairs with scores calculated using the metrics in (ii) that exceed a pre-specified cutoff; if a record $i$ has multiple possible matches $j$ that exceed the cut-off, then choose the match with the highest score (or pick a random match if there is a tie).  
\end{enumerate}

Another record linkage procedure could be defined identically for steps (i)-(iii), but use a probabilistic matching rule that does not enforce one-to-one matching:
\begin{enumerate}
\item[(iv*)]  use the Expectation Maximization algorithm to estimate the probability that each record pair is a match, and then  designate matches�using the Fellegi-Sunter rule for pre-specified tolerances for Type I and Type II errors 
\end{enumerate} 

Except in very specific cases, the estimated $\varphi$ produced by the preceding procedures will be different.  In the first procedure, each $x_i$ associated with at most one matched $y_j$; however, in the second, a single $x_i$�may be associated  with multiple possible $y_j$.  In the latter case, the estimated $\varphi$ is a \textit{correspondence}, which needs to be addressed when deciding how to estimate of $\theta_0$. 

As pointed out by \cite{bailey2017}, record linkage procedures differ according to the set of assumptions that motivate their use.  However, all of the procedures discussed in this paper will be studied under a unifying set of assumptions:

%Bailey: ``In summary, existing linking methods involve multiple differing assumptions that may affect match
%rates but have unknown effects on linking errors. Which set of assumptions should researchers use in
%different contexts? What are the implications of using variations on these algorithms? This paper seeks to
%answer these questions by presenting a systematic comparison of methods in different records and periods."


\begin{enumerate}
\item (De-duplication) Within a given dataset, each observation refers to a distinct entity.  That is, if two observations share the same identifier, they represent two different individuals.
\item (No unobserved sample selection) The observed $x_i$ and $y_j$ are random samples conditional on $w_i$ and $w_j$, respectively.  This means that all individuals with the same identifying information have equal probability of appearing in the sample. 
\item There exists a unique $\theta_0 \in \Theta$ that satisfies the relationship in (\ref{model}), that can be consistently estimated using standard econometric techniques if $\varphi_0$ is known.
\end{enumerate}

The next section discusses in detail the exact record linkage techniques that will be studied, and the corresponding assumptions.

\section{Overview of matching techniques}
In statistics, techniques for recovering $\varphi$ are called \textit{record linkage} procedures; but for the purposes of this paper, I will use the term ``matching procedures" interchangeably.   

In the section that follows I talk about each of these. 





  which records to compare  data pre-processing, blocking/implementation, and match assignment.  Data pre-processing includes selecting and standardizing variables for matching (by applying phonetic algorithms to string variables for example), and defining a distance metric for string variables.  Implementation decisions include blocking rules to limit the number of comparisons for computational feasibility and picking relative weights for disagreements in different variables.   Finally, the review process involves designating record pairs as matches if a one-to-one matching is desired (or defining a cut-off for multiple matches). 

Matching procedures are either deterministic or probabilistic.   Deterministic methods are those like \cite{abe} and \cite{ferrie}, where a fixed set of rules determine which records are matching and which are not.  On the other hand, probabilistic methods attempt to ``let the data speak", at the cost of computational feasibility, as they involve estimating the probability that each record pair within a pre-specified block refer to a match.  In many instances, it is not clear which perform better (especially when the datasets are large!)


record linkage process is x y,z, 










There are 44,000 files in Dataset A.  Dataset B consists of many, many million observations.  The goal is to find the matches for the 44,000 individuals... this is not an insignificant task.  Blocking is crucial. 
Although individuals with distinct names are unlikely to appear on multiple death records, individuals with common names like ``John Smith" may be linked to multiple death records.  Similarly, if the econometrician matches individuals using a subset of the variables in $w_i$ (as in the case that females are matched by first name only, and place and year of birth), she is likely to find many possible links that are equally credible.  

The following note formalizes the assumptions that are necessary to estimate the model ($\ref{model}$) without dropping observations with non-unique matches.


\textbf{Assumption 1.} The observed $(x_i, z_i, w_i)$ is a random sample drawn from the marginal distribution $f(x^*, z^*, w^*)$.   The $\{y_{i\ell}\}_{\ell=1}^{L_i}$ is a random sample drawn from $f(y^* | w_i, L_i)$, so that $(x_i, z_i) = (x_i^*,z_i^*)$ and $y_{i\ell}$ are independent conditional on $w_i$ for $y_{i\ell} \neq y_i^*$. 

\textbf{Assumption 2.}  There is exactly one $y_{i\ell} = y_i^*$, $\ell\in \{1, \dots, L_i\}$ for all $i.$ That is, we assume that one of the $\{y_{i\ell}\}_{\ell=1}^{L_i}$ is drawn from the marginal distribution $f(y_i^* | x_i^*, z_i^*) = f(y_i^* | x_i, z_i)$.   

This assumption implies we can write $y_i^* = \sum_{\ell=1}^{L_i} s_{i\ell} y_{i\ell}$, where $s_{i\ell}$ is an unobserved latent variable that equals 1 if $(y_{i\ell}, x_i, z_i) = (y_i^*, x_i^*, z_i^*)$, and that equals 0 otherwise.  
Also, $\sum_{\ell=1}^{L_i} s_{i\ell} = 1$ for all $i$, and we can rewrite \ref{model} as,
\begin{equation} E\left[m(y_i^*, x_i^*, z_i^*)\right] = 0 \iff E\left[m(y_{i\ell},x_i,z_i; \theta_0) | s_{i\ell} =1 \right] = 0  \tag{$1^*$} \label{truemodel} \end{equation} 

\textbf{Assumption 3}. The identifying variables $w_i$ in $(x_i, z_i, w_i)$ exactly match, or are sufficiently close to $w_i$ in $(\{y_{i\ell}\}_{\ell=1}^{L_i}, w_i)$, such that the researcher cannot distinguish which $y_{i\ell} = y_i^*$ for $\ell \in \{1, \dots, L_i\}$.  In other words, the researcher behaves as if,  
$$P(s_{i\ell}=1 | w_i, L_i) = \frac{1}{L_i}$$ 
Assumptions 1 and 3 rule out unobserved sample selection, in the sense that all individuals with the same identifying information have equal probability of appearing in the sample.  This assumption would be violated if, for example, higher income individuals have a greater probability of appearing in the sample, unless $w_i$ includes income.   




\bibliography{./proposal_bib} 
\bibliographystyle{IEEEtranN}
\newpage
\section{Matching Methods}

For the purposes of this paper, I define a record linkage procedure as a set of rules about data pre-processing, blocking/implementation, and match assignment.  Data pre-processing includes selecting and standardizing variables for matching (by applying phonetic algorithms to string variables for example), and defining a distance metric for string variables.  Implementation decisions include blocking rules to limit the number of comparisons for computational feasibility and picking relative weights for disagreements in different variables.   Finally, the review process involves designating record pairs as matches if a one-to-one matching is desired (or defining a cut-off for multiple matches). 

Matching procedures are either deterministic or probabilistic.   Deterministic methods are those like \cite{abe} and \cite{ferrie}, where a fixed set of rules determine which records are matching and which are not.  On the other hand, probabilistic methods attempt to ``let the data speak", at the cost of computational feasibility, as they involve estimating the probability that each record pair within a pre-specified block refer to a match.  In many instances, it is not clear which perform better (especially when the datasets are large!)

In this paper I do two implementations of each type (not exhaustive!).   X YZ already review different types.   


\begin{figure}[h!]
\centering
\caption{Overview of matching methods}
\includegraphics[width=0.7\textwidth]{./RecordLinkageGraphics.pdf}
\end{figure}

I will do 4: ABE Code, FErrie (done already), ABE R CODE, AND FASTLINK. 

INSERT A GRAPHIC WITH THE METHODS I WILL TEST



TABLE WITH SURVEY OF MATCHING METHODS used in literature.   categoreis are like enforces one to one matching, blocking, string comparator, phonetic algorithm (this is lit review essentially/annotated bibliography) 
\begin{itemize}
\item Deterministic:  Rely heavily on phonetic algorithms (i.e. SOUNDEX, NYSIIS, Metaphone, Spanish Metaphone) and string distance measures (SPEDIS, Jaro-Winkler).    \cite{abe2019} discuss differences in these choices and call SOUNDEX outdated.
\begin{enumerate}
\item \cite{aizer2016}:  Described in appendix, match uses first name, middle initial, last name, day, month and years of birth.   Match allows for errors in strings (using SOUNDEX and SPEDIS) and in single digits for DOB. 
\item Abramtizky, Boustan and Eriksson (ABE) algorithms -- discard observations that do not have unique matches; they typically perform multiple variations of the same algorithm to check robustness.  Perform matching in both directions, and then take the intersection of the two matched samples.  Implemented with abematch stata command. 
\item ABE-JW adjustment algorithm -- block by place of birth, first letter of first and last names match.  Jaro-winkler can be implemented with stata jarowinkler command; R package stirngdist.  They also only accept one-to-one matching. 

Limitation of ABE algorithms is that ``it is not clear how to
appropriately weight differences in name spelling versus differences in age when comparing two records".  
\end{enumerate}
\item Probabilistic  (see Winkler 2006 for survey)
\begin{itemize}
\item E-M Algorithm \cite{arp2018} still blocking by same place of birth, predicted year of birth, and first letter of first and last name.  Still use decision rule to enforce one-to-one matching. 
\item Training sample (Ruggles and Feigenbaum) -- trained to NOT allow mutliple matches; generates a predicted probability of being a match for each pair of records in A and B
\item IPUMS linking method:  trains support vector machine on training sample of manually classified records (like Feigenbaum 2016)  In historical applications this is problematic due to sample attrition.  The DGP changes, so a full likelihood is a good idea. 
\end{itemize}
\end{itemize}
- Overview of matching methods

Important measurements:  estimated type 1, type 2 errors; representativeness of sample, sample size, overlapping of samples 
- Comparison of matching methods from (a) theoretical perspective, (b) with simulated data, (c) with actual data
\begin{enumerate}
\item  Estimation Methods 
\begin{itemize}
\item Anderson, Honore, Lleras-Muney (2019)
\item Lahiri Larsen
\item Scheuren Winkler 
\end{itemize}
- Overview of estimation methods


- Comparison of estimation methods from (a) theoretical perspective, (b) with simulated data, (c) with actual data

(3) Further investigation/follow-up simulations inspired by steps 1 and 2  
\end{enumerate}



% In contrast to deterministic record linkage, which has been used extensively in the economics literature (Ferrie, cite etc.).... descriptions here

% Thus, it is the goal of this paper.  



% In the literature, there are different methods for matching:  deterministic matching methods, such as those used by Joe Ferrie, probabilistic record linkage, and machine learning techniques.   Using the same datasets (simulated and real), I will apply each of these methods in order to create different matched versions of the same datasets. 

% The output of the matching algorithms are different.  In the case of deterministic matching, the output is a configuration of the data.  However, probabilistic and machine learning matching also output estimated probabilities of matches, which are then available for subsequent inference.   One question that AHL examine is how using information about match quality can improve the quality of the estimators, in terms of MSE reduction.   


Data problems in practice \cite{bailey2017}
\begin{itemize}
\item ``time-invariant" features not time-invariant -- names misspelled, individual reported incorrectly, or individual changed name 
\item Age heaping -- rounding ages to the nearest multiple of five
\item digitization of handwritten manuscripts 
\end{itemize}

String comparators: NYSIIS, Soundex, Jaro-Winkler, Levenstein, etc. 




\section{Annotated bibliography}
\begin{itemize}
\item Neter, Maynes, and Ramanathan (1965): small mismatch errors in finite population sampling can lead to a substantial bias in estimating the relationship between response errors and true values
\item Scheuren and Winkler (1993): propose method for adjusting for bias of mismatch error in OLS
\item SW (1997, 1991): iterative procedure that modifies regression and matching results for apparent outliers 
\item Lahiri and Larsen (2005):   provides unbiased estimator directly instead of bias correction for OLS, by applying regression to transformed model 
\item Abramitzky, Mill, P\'erez (2019): guide for researchers in the choice of which variables to use for linking, how to estimate probabilities, and then choose which records to use in the analysis.  Created R code and stata command to implement the method
\item Ferrie 1996, Abramitzky, BOustan and Eriksson (2012 2014 2017) are deterministic.  Conservative methods require no other potential match with same name within a 5-year band
\item Semi-automated Feigenbaum, Ruggles et al 
\item Abramitzky, Boustan, Eriksson, Feigenbaum, P\`erez (2019): evaluate different automated methods for record linkage, specifically deterministic (like Ferrie and ABE papers), machine learning Feigenbaum approach, and the AMP approach with the EM algorithm.  Document a frontier between type I and type II errors; cost of low false positive rates comes at cost of designating relatively fewer (true) matches.  Humans typically match more at a cost of more false positives.  They study how different linking methods affect inference -- sensitivity of regression estimates to the choice of linking algorithm.  They find that the parameter estimates are stable across linking methods.  Find effect of matching algorithm on inference is small. 
\item Bailey et al. (2017) review literature on historical record linkage in US and examines performance of automated record linkage algorithms with two high-quality historical datasets and one synthetic ground truth.  They conclude that no method consistently produces representative samples; machine linking has high number of false links and may introduce bias into analyses.  

Treatment of equally likely -- equal probability weighting of tied candidates (Bleakley and Ferrie 2016); weighted combo of linking features to ehlp disambiguate potential matches.  Ferrie 96; Old ABE, new ABE, and Feigenbaum. 


\item Survey paper from handbook of econometrics
\item For example, Goeken et al. (2017) document that
in two enumerations of St. Louis in the 1880 Census, nearly 46 percent of first names are not exact matches,
and the Early Indicators project notes that 11.5 percent of individuals in the Oldest Old sample have a
shorter first name in pension records than in the original Civil War enlistment records (Costa et al. 2017). 
\end{itemize}

Overall, high variability in performance of matching methods depending on choice of variables, string comparators used. 


\section{Empirical Application}



\subsection{Data}

What are the long-run effects of cash transfers to poor families?  Specifically, are there lifelong benefits for children raised in poor families?  Transfers may not help poor children if the amounts are insufficent, parents mis-allocate funds, or the transfers induce behavioral changes that are detrimental to the child.  

Evaluating whether cash transfers improve outcomes necessitates identifying a plausible counterfactual.  \cite{aizer2019}  collect administrative records from the MOthers' Pension program (1911-1935), which was the first US government-sponsored welfare program for poor mothers with dependent children.  The intent of the MP program was to imrpove the conditions of ``young children that have become dependent through the loss or disability of the breadwinner" (Abbott 1993, p 1).  The transfers generally represented 12-25 percent of family income, and typically lasted for three years.   

The authors measure the impact in terms of longetivity and other outcomes of the children whose mothers applied to the program.  These data include information on thousands of accepted and rejected applciants born between 1900 and 1925, most of whom had died by 2012.  The idetnifying information in the application records allows them to link children with other datasets to trace their lifetime outcomes.

Identification comes from comparing children of mothers who applied for transfers and who were initially deemed eligible, but were denied upon further investigation.  This is a standard strategy in program evaluation that has been used successfully in studies of disability insurance.  The validity of this strategy hinges on the assumpotion that accepted and rejected mothers and their children do not differ on unobservable characteristics.  Rejected mothers were on average slightly better-off, based on observable characteristics at the time of the applciation.    Authors say that the outcomes for boys of rjeected mothers provide a best-case scenario (upper bound) for what could be expected of beneficiaries in the absence of transfers.  

Data collected on over 16,000 boys from 11 states who were born between 1900 and 1925, and whose mothers applied to the MP program; find that transfers increased longevity by about 1 year.  Poorest families in the sample had longevity increased by 1.5 years of life.  Results are robust to alternative function form specifications, counterfactual comparisons, and treatment of attrition.  They interpret their results as the effect of cash transfers alone.  

The authors match also a subset of the records to WWII enlistment and 1940 census records; cash transfers reduced the probability fo being underweight by half, increased educational attainment by 0.34 years, and increased income in early adulthood by 14 percent.  They say that these mechanisms are responsible for 75 percent of the increase in longevitiy

\subsection{Empirical Model}

\subsubsection*{Basic empirical model}
\begin{equation}
\log(\text{age at death})_{ifts} = \theta_0 +  \theta_1 MP_f + \theta_2 X_{if} + \theta_3 Z_{st} + \theta_c + \theta_t + \epsilon_{if} 
\end{equation}
\subsubsection*{Model to address attrition and mutliple matches}
\begin{equation}
P(\text{survived to age $a = 1$})_{iftcs} = f(\theta_0 + \theta_1 MP_f + \theta_2 X_{if} + \theta_3 Z_{st} + \theta_c + \theta_t + \epsilon_{if}) 
\end{equation}

(i) above using unique matches -- baseline , (ii) above using multiple matches -- ahl baseline, (iii) above in bayesian framework -- bayesian baseline, (iv) above using probabilities of matches (freq + bayesian versions)

Replicate Table 4 with different matching; different estimation techniques 

Do a basic Bayesian log normal model for different states -- pooled v. separate
multiple matching


\end{document}