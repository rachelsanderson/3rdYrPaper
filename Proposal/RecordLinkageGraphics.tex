%%%%%%%%%%%%%%%%%%%%%%%%%%%%%%%%%%%%%%%%%%%%%%%%%%%%%%%%%%%%%%%
%
% Welcome to Overleaf --- just edit your LaTeX on the left,
% and we'll compile it for you on the right. If you open the
% 'Share' menu, you can invite other users to edit at the same
% time. See www.overleaf.com/learn for more info. Enjoy!
%
% Note: you can export the pdf to see the result at full
% resolution.
%
%%%%%%%%%%%%%%%%%%%%%%%%%%%%%%%%%%%%%%%%%%%%%%%%%%%%%%%%%%%%%%%
\documentclass{article}
\usepackage{tikz}
%%%<
\usepackage{verbatim}
\usepackage[active,tightpage]{preview}
\PreviewEnvironment{tikzpicture}
\setlength\PreviewBorder{10pt}%
%%%>
\begin{comment}
:Title: Work breakdown structures aka WBS diagrams
:Tags: Charts;Diagrams;Trees
:Author: Gonzalo Medina
:Slug: work-breakdown-structure

A work breakdown structure (WBS) diagram, is for decomposing a task into smaller parts, which helps organizing and performing. This example diagram shows possible tasks for designing a TikZ diagram. The basis is a tree, nodes were addes below its child nodes. It was originally posted by Gonzalo Medina as TeX.SE (http://tex.stackexchange.com/q/81809/213), modified by Stefan Kottwitz.
\end{comment}
\usetikzlibrary{arrows,shapes,positioning,shadows,trees}

\tikzset{
  basic/.style  = {draw, text width=2cm, drop shadow, font=\sffamily, rectangle},
  root/.style   = {basic, rounded corners=2pt, thin, align=center,
                   fill=white!30},
  level 2/.style = {basic, rounded corners=6pt, thin,align=center, fill=white!60,
                   text width=8em},
  level 3/.style = {basic, thin, align=left, fill=white!60, text width=6.5em}
}

\begin{document}
\begin{tikzpicture}[
  level 1/.style={sibling distance=40mm},
  edge from parent/.style={->,draw},
  >=latex]

% root of the the initial tree, level 1
\node[root] {Record Linkage}
% The first level, as children of the initial tree
  child {node[level 2] (c1) {Deterministic}}
  child {node[level 2] (c2) {Probabilistic}};

% The second level, relatively positioned nodes
\begin{scope}[every node/.style={level 3}]
\node [below of = c1, xshift=15pt] (c11) {Ferrie (1996)};
\node [below of = c11] (c12) {ABE (2016)};
\node [below of = c12] (c13) {One more};

\node [below of = c2, xshift=15pt] (c21) {AMP};
\node [below of = c21] (c22) {Machine learning};
\node [below of = c22] (c23) {fastLink};
\node [below of = c23] (c24) {Using overlays};

\end{scope}

% lines from each level 1 node to every one of its "children"
\foreach \value in {1,2,3}
  \draw[->] (c1.195) |- (c1\value.west);

\foreach \value in {1,...,4}
  \draw[->] (c2.195) |- (c2\value.west);

% \foreach \value in {1,...,5}
%   \draw[->] (c3.195) |- (c3\value.west);
\end{tikzpicture}
\end{document}