\documentclass[11pt]{amsart}
\usepackage{geometry}                % See geometry.pdf to learn the layout options. There are lots.
\geometry{letterpaper}                   % ... or a4paper or a5paper or ... 
%\geometry{landscape}                % Activate for for rotated page geometry
%\usepackage[parfill]{parskip}    % Activate to begin paragraphs with an empty line rather than an indent
\usepackage{graphicx}
\usepackage{amssymb}
\usepackage{amsmath}
\usepackage{epstopdf}
\usepackage{setspace}
\DeclareGraphicsRule{.tif}{png}{.png}{`convert #1 `dirname #1`/`basename #1 .tif`.png}
\newcommand{\Var}[1]{\text{Var}\left(#1\right)}
\title{Brief Article}
\author{The Author}
%\date{}                                           % Activate to display a given date or no date

\begin{document}
%\section{}
%\subsection{}
\doublespacing

\section{Optimality of the AHL Estimator}

% To show optimality of the AHL consider the possibility of incorporating probabilities of matches in the AHL framework.  The interesting result is that the minimum variance unbiased estimator depends on the ratio of the structural error in the model to the variance in the reduced form estimation for $y$, which also depends on the precision of the estimator $\hat{g}$.  This is an interesting result because it connects to the Horwitz-Thompson estimator, and work on inverse propensity score weighting. 

Consider the problem of estimating the mean of a random variable $X \sim F_X(\mu; \sigma^2)$ using two observations $X_{1}$ and $X_{2}$.  With probability $\pi$, $X_1$ is drawn from the true distribution $F_X$ and $X_2$ is noise drawn from the distribution $F_Y(\kappa, \omega^2)$.  With probability $1-\pi$, $X_2$ is drawn from the correct distribution and $X_1$ is noise.  Under this specification, exactly one of $X_1$ or $X_2$ is drawn from the distribution of interest at all times.  

Observe that if $\pi$ is known, we can construct an unbiased estimator using only $X_1$,
\begin{equation}
\hat{\mu}_1 = \frac{X_1}{\pi} - \frac{1-\pi}{\pi} \kappa 
\label{mu1}
\end{equation} 

Similarly, we can construct an unbiased estimator using only $X_2$,
\begin{equation}\hat{\mu_2} = \frac{X_2}{1-\pi} - \frac{\pi}{1-\pi} \kappa \label{mu2} \end{equation}

Compare these to an estimator that uses both $X_1$ and $X_2$, 
\begin{equation}
\hat{\mu} = a_1X_{1} + a_2 X_2 -  a_3 \kappa \label{mu} \end{equation}
which has the following expectation,
$$ E[\hat{\mu}] = (a_1 \pi + a_2 (1-\pi))\mu + (a_1(1-\pi)+a_2\pi - a_3)\kappa $$
so that unbiased, requires
\begin{gather}
    a_1\pi + a_2(1-\pi) = 1 \implies a_2(a_1) = \frac{1}{1-\pi} - \frac{a_1 \pi}{1-\pi} \\
    a_1 (1-\pi) + a_2 \pi = a_3 \implies a_3(a_1) = \frac{\pi}{1-\pi}  + \frac{a_1 - 2a_1 \pi}{1-\pi} 
\end{gather}
Hence we can write $\hat{\mu}$ as a function of $a_1$,  
 $$\hat{\mu}(a_1) = a_1 X_1 +  \left(\frac{1}{1-\pi} - \frac{a_1 \pi}{1-\pi}\right)  X_2 - \left(\frac{\pi}{1-\pi} + \frac{a_1 - 2a_1\pi}{1-\pi}\right)\kappa $$ 
When $a_1 = \frac{1}{\pi}$, then $\hat{\mu} = \hat{\mu}_1$; and if $a_1 = 0$ then $\hat{\mu}=\hat{\mu_2}$. 

We can write:
\begin{align*}
\hat{\mu} &= (a_1 \pi) \hat{\mu}_1 + (1-a_1\pi) \hat{\mu}_2 + a_1(1-\pi)\kappa  + \frac{\pi}{1-\pi}(1-a_1)\kappa - \left(\frac{\pi}{1-\pi} + \frac{a_1 - 2a_1\pi}{1-\pi}\right)\kappa  \\
&=  (a_1 \pi) \hat{\mu}_1 + (1-a_1\pi) \hat{\mu}_2 - (a_1\pi)\kappa
\end{align*}

Hence any unbiased estimator $\hat{\mu}$ that uses $X_1$ and $X_2$ can be written as a linear combination of estimators using only $X_1$ or $X_2$.  The problem of finding the minimum variance, unbiased estimator $\hat{\mu}$ reduces to finding $d^*$ that solves 
$$\min_d\  \Var{d \hat{\mu}_1 + (1-d) \hat{\mu}_2}$$
which is solved by $d^*= 0$ or $d^* = 1$ depending on whether $\Var{\hat{\mu}_1}$ or $\Var{\hat{\mu}_2}$ is smaller.  

I now show that $\Var{\hat{\mu}_1} > \Var{\hat{\mu}_2}$, without loss of generality, except when $\pi = 0.5$, or $\sigma^2 = \omega^2 = (\mu-\kappa)^2$.  Observe that,
\begin{align*} \Var{\hat{\mu}_1} &=  \frac{\text{Var}(X_1)}{\pi^2} = \frac{1}{\pi^2}\left(\pi \sigma^2 + (1-\pi) \omega^2 + \pi(1-\pi)(\mu-\kappa)^2\right) \\
\Var{\hat{\mu}_2} &= \frac{\Var{X_2}}{(1-\pi)^2} = \frac{1}{(1-\pi)^2}\left((1-\pi)\sigma^2 + \pi \omega^2 + \pi(1-\pi)(\mu-\kappa)^2\right) 
\end{align*}

This follows from the law of total variance, with the random variable $D = 1$ if $X_1$ is drawn from the correct distribution (and $X_2$ is drawn from the incorrect distribution), and $D=0$ otherwise. 
\begin{align*} \Var{X_1} &= E[\Var{X_1 | D}] + \Var{E[X_1| D]} \\ 
&= P(D=1)\sigma^2 +  P(D=0)\omega^2 + \Var{\mu D + \kappa (1-D)} \\
&= \pi \sigma^2 + (1-\pi) \omega^2 + \pi(1-\pi)(\mu-\kappa)^2
\end{align*}
Similarly,
$$\Var{X_2} = (1-\pi)\sigma^2 + \pi \omega^2 + \pi(1-\pi)(\mu-\kappa)^2 $$ 
Thus, $\Var{\hat{\mu}_1}$ and $\Var{\hat{\mu}_2}$ can be written as functions of $\sigma^2, \omega^2,$ and $(\mu-\kappa)^2$,
\begin{align*} 
g(\sigma^2, \omega^2, (\mu-\kappa)^2, x) &\equiv  \frac{1}{x^2}\left(x \sigma^2 + (1-x)\omega^2 + x (1-x) (\mu -\kappa)^2\right)\\
\Var{\hat{\mu}_1} &= g(\sigma^2, \omega^2, (\mu-\kappa)^2, \pi) \\
\Var{\hat{\mu}_2} &= g(\sigma^2, \omega^2, (\mu-\kappa)^2, 1-\pi) 
\end{align*}
Importantly, 
$$\frac{\partial g(\sigma^2, \omega^2, (\mu-\kappa)^2, x)}{\partial x} = \frac{\omega^2 (x-2) - x (\sigma^2 + (\mu-\kappa)^2)}{x^3}< 0,  \ x \in (0,1)$$
and so $\Var{\hat{\mu}_{\ell}}$ is strictly decreasing in the probability that the observation $\ell$ is drawn from the correct distribution, and $\Var{\hat{\mu}_1} \neq \Var{\hat{\mu}_2}$ unless $\pi = 0.5$.   Thus, the minimum variance unbiased estimator is equal to $\hat{\mu}_{\ell}$ for the observation $\ell$ that has the highest probability of being correct.

The above result holds also for $L$ observations $X_1, \dots, X_{L}$ with corresponding probabilities $\pi_1,\dots, \pi_L$; that is, the minimum variance unbiased estimator $\hat{\mu}$ will use only $X_{\ell}$ with the highest $\pi_{\ell}$ and apply inverse probability weighting. 

Now consider a sample of $N$ sets of i.i.d. observations, $\left\{\{X_{i\ell}\}_{\ell=1}^{L_i}\right\}_{i=1}^N$.  If the values $$\pi_{i\ell} = \text{Pr}(X_{i\ell}\text{ is drawn from the correct distribution}),   \sum_{\ell=1}^{L_i} \pi_{i\ell} = 1$$ are known for all $i$, then the optimal estimator is 
$$\hat{\mu} = \frac{1}{N} \sum_{i=1}^N \hat{\mu}_i  = \frac{1}{N}\sum_{i=1}^N \frac{X_{i\ell_i}}{\pi_{i\ell_i}} - \frac{1-\pi_{i\ell_i}}{\pi_{i\ell_i}}\kappa$$
where $\hat{\mu}_i$ is the minimum variance estimator constructed for observations $\{X_{i\ell}\}_{\ell=1}^{L_i}$. 

Since $\hat{\mu}$ is an inverse probability weighting estimator, small values of $\pi_{i\ell}$ may be detrimental for its finite sample performance.  Rather than dropping observations whose maximal $\pi_{i\ell}$ is small, however, it is possible to give these observations equal weights for all $\{X_{i\ell}\}$, as in the AHL estimator.  I hypothesize that if $\pi_{i\ell} < 0.5$ (or another threshold) for all $\ell$, then it is better to give equal weights to all $X_{i\ell}$ associated with $i$, even when the $\pi_{i\ell}$ are known.

In practice, $\pi_{i\ell}$ and $\kappa$ will most likely need to be estimated. Imprecise estimates of $\hat{\pi}_{i\ell}$ may introduce large finite-sample bias, as can seen via simulation.  Similarly the variance of $\hat{\kappa}$ may not be insignificant.   The advantage of the AHL estimator is that it does not require knowledge about $\pi_{i\ell}$, and may even have best worst case performance in terms of asymptotic MSE; however, the bias correction term $\hat{\kappa}$ still needs to be estimated. 

\end{document}  