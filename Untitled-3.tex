\documentclass[11pt]{amsart}
\usepackage{geometry}                % See geometry.pdf to learn the layout options. There are lots.
\geometry{letterpaper}                   % ... or a4paper or a5paper or ... 
%\geometry{landscape}                % Activate for for rotated page geometry
%\usepackage[parfill]{parskip}    % Activate to begin paragraphs with an empty line rather than an indent
\usepackage{graphicx}
\usepackage{amssymb}
\usepackage{amsmath}
\usepackage{epstopdf}
\usepackage{setspace}
\DeclareGraphicsRule{.tif}{png}{.png}{`convert #1 `dirname #1`/`basename #1 .tif`.png}
\newcommand{\Var}[1]{\text{Var}\left(#1\right)}
\title{Brief Article}
\author{The Author}
%\date{}                                           % Activate to display a given date or no date

\begin{document}
\maketitle
%\section{}
%\subsection{}
\doublespacing

\section{Optimality of the AHL Estimator}

% To show optimality of the AHL consider the possibility of incorporating probabilities of matches in the AHL framework.  The interesting result is that the minimum variance unbiased estimator depends on the ratio of the structural error in the model to the variance in the reduced form estimation for $y$, which also depends on the precision of the estimator $\hat{g}$.  This is an interesting result because it connects to the Horwitz-Thompson estimator, and work on inverse propensity score weighting. 

Consider the problem of estimating the mean of a random variable $X \sim F_X(\mu; \sigma^2)$ using two observations $X_{1}$ and $X_{2}$.  With probability $\pi$, $X_1$ is drawn from the true distribution $F_X$ and $X_2$ is noise drawn from the distribution $F_Y(\kappa, \omega^2)$.  With probability $1-\pi$, $X_2$ is drawn from the correct distribution and $X_1$ is noise.  Under this specification, exactly one of $X_1$ or $X_2$ is drawn from the distribution of interest at all times.  

Observe that if $\pi$ is known, we can construct an unbiased estimator using only $X_1$,
\begin{equation}
\hat{\mu}_1 = \frac{X_1}{\pi} - \frac{1-\pi}{\pi} \kappa 
\label{mu1}
\end{equation} 

Similarly, we can construct an unbiased estimator using only $X_2$,
\begin{equation}\hat{\mu_2} = \frac{X_2}{1-\pi} - \frac{\pi}{1-\pi} \kappa \label{mu2} \end{equation}

Compare these to an estimator that uses both $X_1$ and $X_2$, 
\begin{equation}
\hat{\mu} = a_1X_{1} + a_2 X_2 -  a_3 \kappa \label{mu} \end{equation}
which has the following expectation,
$$ E[\hat{\mu}] = (a_1 \pi + a_2 (1-\pi))\mu + (a_1(1-\pi)+a_2\pi - a_3)\kappa $$
so that unbiased, requires
\begin{gather}
    a_1\pi + a_2(1-\pi) = 1 \implies a_2(a_1) = \frac{1}{1-\pi} - \frac{a_1 \pi}{1-\pi} \\
    a_1 (1-\pi) + a_2 \pi = a_3 \implies a_3(a_1) = \frac{\pi}{1-\pi}  + \frac{a_1 - 2a_1 \pi}{1-\pi} 
\end{gather}


Any unbiased estimator of the form CITE $\hat{\mu}$ can be written as a linear combination of $\hat{\mu}_1$ and $\hat{\mu}_2$ :

$$\hat{\mu} = d_1 \hat{\mu}_1 + d_2 \hat{\mu_2}$$

The variance of $\hat{\mu}$ depends on:

\begin{align*} \Var{\hat{\mu}_1} &=  \frac{\text{Var}(X_1)}{\pi^2} = \frac{1}{\pi^2}\left(\pi \sigma^2 + (1-\pi) \omega^2 + \pi(1-\pi)(\mu-\kappa)^2\right) \\
\Var{\hat{\mu}_2} &= \frac{\Var{X_2}}{(1-\pi)^2} = \frac{1}{(1-\pi)^2}\left((1-\pi)\sigma^2 + \pi \omega^2 + \pi(1-\pi)(\mu-\kappa)^2\right) 
\end{align*}

This follows from the law of total variance, with the random variable $D = 1$ if $X_1$ is drawn from the correct distribution and $D=0$ otherwise. 

\begin{align*} \Var{X_1} &= E[\Var{X_1 | D}] + \Var{E[X_1| D]} \\ 
&= P(D=1)\sigma^2 +  P(D=0)\omega^2 + \Var(\mu D + \kappa (1-D)) \\
&= \pi \sigma^2 + (1-\pi) \omega^2 + \pi(1-\pi)(\mu-\kappa)^2
\end{align*}

and similarly 
$$\Var{X_2} = (1-\pi)\sigma^2 + \pi \omega^2 + \pi(1-\pi)(\mu-\kappa)^2 $$ 

The important observation is that the variances are functions parameterized by $(\sigma^2, \omega^2, (\mu-\kappa)^2)$, and we can write: 

\begin{align*} \Var{\hat{\mu}_1} &= g(\sigma^2, \omega^2, (\mu-\kappa)^2, \pi) \\
\Var{\hat{\mu}_2} &= g(\sigma^2, \omega^2, (\mu-\kappa)^2, 1-\pi)
\end{align*}

They are exactly equal when $\pi = 1-\pi$; otherwise they are quadratic as in the plot below.   But they are exactly symmetric, so that the optimal estimator will always assign $d_1 = 1, d_2= 0$ or $d_1=0, d_2=1$ depending on whether $\pi > 1-\pi$.  However, if $\pi$ is not estimated precisely, the resulting variance explodes.  The most conservative way is by treating $\pi = 0.5$, which gives the estimate of the mean!  Which is the AHL estimator. .

Also $\frac{\partial g}{\partial \pi} < 0$ on $\pi \in [0,1]$ so that there is a single crossing at $\pi=0.5$!!! So unless $\pi = 0.5$ you always weight the more likely match and give it the proper weights.  Could do this for every single $i$ in the sample?  Divide by $\pi_i$ and you get the Horwitz Thompson.    




\end{document}  